\ifgerman{\chapter{Implementierung}}{\chapter{Tool Support}}
\label{ch:toolsupport}

\hint{Most theses in computer science are accompanied by tool support written by the author of the thesis. Such tools enable an empirical evaluation or simply serve as a proof-of-concept. In particular, tools are typically not the ultimate goal in research, but often necessary to evaluate whether proposed concepts solve real problems. Hence, it is common to write about the tool in a dedicated chapter.}

\hint{The tool chapter has several goals. For supervisors, it typically helps to estimate the implementation effort of a thesis and problems faced during development. For other students, the chapter serves as the documentation of the tool support. That is, students that extend the tool support will use this chapter to get an overview on the architecture and learn from failed attempts. As researchers are typically rather interested in concepts or evaluations, this dedicated chapter on tool support helps to remove clutter from other chapters. Nevertheless, researchers may be interested to read why tool support has been build the way it is and why it is build on certain existing tools or libraries. Write the chapter such that it useful for researchers, students, and supervisors.}

\todots

